\documentclass[11pt]{article} \setlength{\oddsidemargin}{0in}
\setlength{\evensidemargin}{0in} \setlength{\textwidth}{6.5in}
\setlength{\parindent}{0in} \setlength{\textwidth}{16cm}
\setlength{\topmargin}{1in} \addtolength{\topmargin}{-1.5in}
\setlength{\textheight}{23cm} \setlength{\parskip}{0.75cm}

% Brackets
\usepackage{mathtools} \DeclarePairedDelimiter\ceil{\lceil}{\rceil}
\DeclarePairedDelimiter\floor{\lfloor}{\rfloor}

% Tikz settings
\usepackage{tikz} \usetikzlibrary{trees} \usetikzlibrary {positioning}
\definecolor {mypurple}{cmyk}{0.6,0.4,0.1,0} \definecolor
{myred}{cmyk}{0,0.3,0.3,0} \usetikzlibrary{fit,shapes.misc}

% Typesetting options
\usepackage{fancyvrb} \usepackage{amsmath,amsfonts,amssymb}
\usepackage [english]{babel} \usepackage [autostyle, english =
american]{csquotes} \usepackage[none]{hyphenat} \usepackage{url}

% Graph
\usepackage{tikz}

\begin{document}

\noindent CSCI 3104 Spring 2018 \hfill Problem Set 7\\
First Last (mm/dd)

\hrulefill

{\fontfamily{cmr}\selectfont

  % ******************* PROBLEM 1 *********************
  \section*{Problem 1}

  \textit{(45 pts) Recall that the string alignment problem takes as
    input two strings $x$ and $y$, composed of symbols
    $x_i , y_j \in \Sigma$, for a fixed symbol set $\Sigma$, and returns a
    minimal-cost set of edit operations for transforming the string
    $x$ into string $y$.}

  \textit{Let $x$ contain $n_x$ symbols, let $y$ contain $n_y$
    symbols, and let the set of edit operations be those defined in
    the lecture notes (substitution, insertion, deletion, and
    transposition).}

  \textit{Let the cost of \textbf{indel} be 1, the cost of
    \textbf{swap} be 13 (plus the cost of the two sub ops), and the
    cost of \textbf{sub} be 12, except when $x_i = y_j$ , which is a
    “no-op” and has cost 0.}

  \textit{In this problem, we will implement and apply three
    functions.}

  \begin{enumerate}
  \item[(i)] \textit{ \texttt{alignStrings(x,y)} takes as input two
      ASCII strings $x$ and $y$, and runs a dynamic programming
      algorithm to return the cost matrix $S$, which contains the
      optimal costs for all the subproblems for aligning these two
      strings.  }

\begin{verbatim}
alignStrings(x,y) :  // x,y are ASCII strings
  S = table of length nx by ny // for memoizing the subproblem costs
  initialize S      // fill in the basecases
  for i = 1 to nx {
    for j = 1 to ny {
      S[i,j] = cost(i,j) // optimal cost for x[0..i] and y[0..j]
  }}
  return S
\end{verbatim}

  \item[(ii)] \textit{\texttt{extractAlignment(S,x,y)} takes as input
      an optimal cost matrix $S$, strings x, y, and returns a vector a
      that represents an optimal sequence of edit operations to
      convert x into y. This optimal sequence is recovered by finding
      a path on the implicit DAG of decisions made by
      \texttt{alignStrings} to obtain the value $S[n_x , n_y]$,
      starting from $S[0, 0]$.}

\begin{verbatim}
extractAlignment(S,x,y) : // S is an optimal cost matrix from alignStrings
  initialize a            // empty vector of edit operations
  [i,j] = [nx,ny]         // initialize the search for a path to S[0,0]
  while i > 0 or j > 0
    a[i] = determineOptimalOp(S,i,j,x,y) // what was an optimal choice?
    [i,j] = updateIndices(S,i,j,a) // move to next position
  }
  return a
\end{verbatim}

    \textit{When storing the sequence of edit operations in
      \texttt{a}, use a special symbol to denote no-ops.}

  \item[(iii)] \textit{\texttt{commonSubstrings(x,L,a)} which takes as
      input the ASCII string x, an integer $1 \le L \le n_x$, and an
      optimal sequence a of edits to x, which would transform x into
      y. This function returns each of the substrings of length at
      least L in x that aligns exactly, via a run of no-ops, to a
      substring in y.}
  \end{enumerate}

  \begin{enumerate}
  \item[(a)] \textit{From scratch, implement the functions
      \texttt{alignStrings}, \texttt{extractAlignment}, and
      \texttt{commonSubstrings}. You may not use any library functions
      that make their implementation trivial. Within your
      implementation of \texttt{extractAlignment}, ties must be broken
      uniformly at random.}

    \textit{Submit (i) a paragraph for each function that explains how
      you implemented it (describe how it works and how it uses its
      data structures), and (ii) your code implementation, with code
      comments.}

    \textit{Hint: test your code by reproducing the \texttt{APE} /
      \texttt{STEP} and the \texttt{EXPONENTIAL} / \texttt{POLYNOMIAL}
      examples in the lecture notes (to do this exactly, you’ll need
      to use unit costs instead of the ones given above).}
    \\\\
    TODO
    \\
  \item[(b)] \textit{Using asymptotic analysis, determine the running
      time of the call \texttt{commonSubstrings(x, L,
        extractAlignment( alignStrings(x,y), x,y ) )} Justify your
      answer.}
    \\\\
    TODO
    \\
  \item[(c)] \textit{(15 pts extra credit) Describe an algorithm for
      counting the number of optimal alignments, given an optimal cost
      matrix $S$. Prove that your algorithm is correct, and give is
      asymptotic running time.}

    \textit{Hint: Convert this problem into a form that allows us to
      apply an algorithm we’ve already seen.}
    \\\\
    TODO
    \\
  \item[(d)] \textit{String alignment algorithms can be used to detect
      changes between different versions of the same document (as in
      version control systems) or to detect verbatim copying between
      different documents (as in plagiarism detection systems).  The
      two \texttt{data\_string} files for PS7 (see class Moodle)
      contain actual documents recently released by two independent
      organizations. Use your functions from (1a) to align the text of
      these two documents. Present the results of your analysis,
      including a reporting of all the substrings in x of length
      $L = 9$ or more that could have been taken from y, and briefly
      comment on whether these documents could be reasonably
      considered original works, under CU's academic honesty policy.}
    \\\\
    TODO
  \end{enumerate}

  \newpage

  % ******************* PROBLEM 2 *********************
  \section*{Problem 2}

  \textit{(20 pts) Ron and Hermione are having a competition to see
    who can compute the $n$th Pell number $P_n$ more quickly, without
    resorting to magic. Recall that the $n$th Pell number is defined
    as $P_n = 2 P_{n-1} + P_{n-2}$ for $n > 1$ with base cases
    $P_0 = 0$ and $P_1 = 1$.  Ron opens with the classic recursive
    algorithm:}

\begin{verbatim}
Pell(n):
  if n == 0 { return 0 }
  else if n == 1 { return 1 }
  else { return 2*Pell(n-1) + Pell(n-2) }
\end{verbatim}

  \begin{enumerate}
  \item[(a)]{\textit{Hermione counters with a dynamic programming
        approach that “memoizes” (a.k.a.  memorizes) the intermediate
        Pell numbers by storing them in an array $P[n]$. She claims
        this allows an algorithm to compute larger Pell numbers more
        quickly, and writes down the following
        algorithm. \footnote{Ron briefly whines about Hermione's
          \texttt{L[n]=undefined} trick (``an unallocated array!''),
          but she point out that \texttt{MemLuc(n)} can simply be
          wrapped within a second function that first allocates an
          array of size $n$, initializes each entry to undefined, and
          then calls \texttt{MemLuc(n)} as given.}}}

\begin{verbatim}
MemPell(n) {
  if n == 0 { return 0 } else if n == 1 { return 1 }
  else {
    if (P[n] == undefined) { P[n] = 2*MemPell(n-1) + MemPell(n-2) }
    return P[n]
  }
}
\end{verbatim}

    \begin{enumerate}
    \item[i.] \textit{Describe the behavior of \texttt{MemPell(n)} in
        terms of a traversal of a computation tree. Describe how the
        array \texttt{P} is filled.}
    \item[ii.] \textit{Determine the asymptotic running time of
        \texttt{MemPell}. Prove your claim is correct by induction on
        the contents of the array.}
    \end{enumerate}
    TODO
    \\
  \item[(b)] \textit{Ron then claims that he can beat Hermione’s
      dynamic programming algorithm in both time and space with
      another dynamic programming algorithm, which eliminates the
      recursion completely and instead builds up directly to the final
      solution by filling the P array in order. Ron's new algorithm
      \footnote{Ron is now using Hermione’s undefined array trick;
        assume he also uses her solution of wrapping this function
        within another that correctly allocates the array.} is}

\begin{verbatim}
DynPell(n) :
  P[0] = 0, L[1] = 1
  for i = 2 to n { P[i] = 2*P[i-1] + P[i-2] }
  return P[n]
\end{verbatim}

    Determine the time and space usage of \texttt{DynPell(n)}. Justify
    your answers and compare them to the answers in part (2a).
    \\\\
    TODO
    \\
  \item[(c)] \textit{With a gleam in her eye, Hermione tells Ron that
      she can do everything he can do better: she can compute the
      $n$th Pell number even faster because intermediate results do
      not need to be stored. Over Ron’s pathetic cries, Hermione says}

\begin{verbatim}
FasterPell(n) :
  a = 0, b = 1
  for i = 2 to n
    c = 2 * a + b
    a = b
    b = c
  end
  return a
\end{verbatim}

    Ron giggles and says that Hermione has a bug in her
    algorithm. Determine the error, give its correction, and then
    determine the time and space usage of
    \texttt{FasterPell(n)}. Justify your claims.
    \\\\
    TODO
    \\
  \item[(d)] \textit{In a table, list each of the four algorithms as
      columns and for each give its asymptotic time and space
      requirements, along with the implied or explicit data structures
      that each requires. Briefly discuss how these different
      approaches compare, and where the improvements come from. (Hint:
      what data structure do all recursive algorithms implicitly
      use?)}
    \\\\
    TODO
    \\

  \item[(e)] \textit{(5 pts extra credit) Implement
      \texttt{FasterPell} and then compute $P_n$ where n is the
      four-digit number representing your MMDD birthday, and report
      the first five digits of $P_n$. Now, assuming that it takes one
      nanosecond per operation, estimate the number of years required
      to compute $P_n$ using Ron’s classic recursive algorithm and
      compare that to the clock time required to compute $P_n$ using
      FasterPell.}
    \\\\
    TODO
  \end{enumerate}
  % ---------------------------------------------------
\end{document}