\documentclass[12pt]{article} \setlength{\oddsidemargin}{0in}
\setlength{\evensidemargin}{0in} \setlength{\textwidth}{6.5in}
\setlength{\parindent}{0in} \setlength{\parskip}{\baselineskip}

\usepackage{amsmath,amsfonts,amssymb,hyperref}
\hypersetup{
    colorlinks=true,
    linkcolor=green,
    filecolor=magenta,      
    urlcolor=blue,
}

\begin{document}

CSCI 3104 Spring 2018 \hfill Problem Set 2\\
Cole Schlisner (01/30)

\hrulefill

\begin{enumerate}

\item \textit{Solve the following recurrence relations using any of
    the following methods: unrolling, tail recursion, recurrence tree
    (include tree diagram), or expansion.}

  \textit{Each each case, show your work.}
  
  \begin{enumerate}
  \item \textit{$T(n)=T(n-2)+Cn$ if $n>1$, and $T(n)=C$ otherwise}
    
    \textbf{Unrolling:} \\

    % 1A
    \begin{align}
    T(n) &= T(n-2) + Cn \\
    &= (T(n-4) + Cn) + Cn \\
    &= ((T(n-6) + Cn) + Cn) + Cn \\
    &= (((T(n-8) + Cn) + Cn) + Cn) + Cn \\
    ... \\
    &= T(n-2i) + iCn
    \end{align}

    \textbf{Base Case}: $n-2i = 1 \implies n = 2i+1 \implies i = \frac{1-n}{2}$ \\
    \medskip
    \textbf{Given} $i = \frac{1-n}{2}$ : \\
    \begin{align}
    T(n) &= T(n-2i) + iCn \\
    &= T(1) + \frac{1-n}{2}Cn \\
    &= C + \frac{1-n}{2}Cn \\
    &= \Theta(n)
    \end{align}
    % 
  
  \newpage 

  \item \textit{$T(n)=3T(n-1)+1$ if $n>1$ and $T(1)=3$}

    \textbf{Unrolling:} \\

    % 1B
    \begin{align}
      T(n) &= 3T(n-1) + 1 \\
      &= 3(3T(n-2) + 1) + 1 \\
      &= 3(3(3T(n-3) + 1) + 1) + 1 \\
      ... \\
      &= 3^iT(n-i) + i
    \end{align}

    \textbf{Base Case}: $n-i = 1 \implies i=n-1$ \\
    \medskip
    \textbf{Given} $i = n-1$ : \\
    \begin{align}
      T(n) &= 3^iT(n-i)+i \\
      &= 3^{n-1}T(1)+(n-1) \\
      &= 3^{n-1}3+n-1 \\
      &= 3^n + n - 1 \\
      &= \Theta(3^n)
    \end{align}
  
  \newpage

  \item \textit{$T(n)=T(n-1)+3^n$ if $n>1$ and $T(1)=3$}

    \textbf{Unrolling:} \\

    % 1C
    \begin{align}
      T(n) &= T(n-1) + 3^n \\
      &= T(n-2) + 3^n + 3^n \\
      &= T(n-3) + 3^n + 3^n + 3^n \\
      ... \\
      &= T(n-i) + i3^n
    \end{align}

    \textbf{Base Case}: $n-i = 1 \implies i=n-1$ \\
    \medskip
    \textbf{Given} $i = n-1$ : \\
    \begin{align}
      T(n) &= T(n-i) + i3^n \\
      &= T(1) + (n-1)3^n \\
      &= n3^n-3^n+3 \\
      &= \Theta(n3^n)
    \end{align}


  \newpage 

  \item \textit{$T(n)=T(n^{1/4}+1)$ if $n>2$, and $T(n)=0$ otherwise}

    \textbf{Unrolling:} \\

    % 1C
    \begin{align}
      T(n) &= T(n^{(\frac{1}{4})}) + 1 \\
      &= T(n^{(\frac{1}{16})}) + 1 + 1 \\
      &= T(n^{(\frac{1}{64})}) + 1 + 1 + 1 \\
      ... \\
      &= T(n^{(\frac{1}{4^i})}) + i
    \end{align}

    \textbf{Base Case}: \\
    \begin{align}
    n^{(\frac{1}{4^i})} &= 2 \\
    \frac{1}{4^i}lg(n) &= lg(2) \\
    lg(n) &= 4^ilg(2) \\
    4^i &= lg(n) \\
    ilg(4) &= lg(lg(n)) \\
    i &= \frac{lg(lg(n))}{2}
    \end{align}
    \medskip
    \textbf{Given} $i = \frac{lg(lg(n))}{2}$ : \\
    \begin{align}
    T(n) &= T(n^{\frac{1}{4^i}}) + i \\
    &= T(n^{\frac{1}{4^{(\frac{lg(lg(n))}{2})}}}) + \frac{lg(lg(n))}{2} \\
    &= T(2) + \frac{lg(lg(n))}{2} \\
    &= \frac{lg(lg(n))}{2} \\
    &= \Theta(lg(lg(n)))
    \end{align}
    
  \end{enumerate}
  
  \newpage
  
\item \textit{Consider the following function:}
  

\begin{verbatim}
    def foo(n) {
        if (n > 1) {
        print( ''hello'' )
        foo(n/3)
        foo(n/3)
    }}
\end{verbatim}

  \textit{In terms of the input n, determine how many times is
    ``hello'' printed. Write down a recurrence and solve using the
    Master method.}

  \medskip

  The recursion tree is a binary tree with each node having input $\frac{n}{3}$ of the parent. The recursion stops when the input $n \leq 1$. 
  \medskip

  Hence the base case is: $\frac{n}{3^i} = 1 \implies i = log_3(n)$ = depth of tree. 
  "Hello" is printed for each node in this tree, and since it is a full binary tree, there are $2^i$ nodes on the i-th level.
  Following this, the total number of nodes in the tree excluding the bottom layer (where n = 1) is given by: $$\sum_{i=0}^{log_3(n)-1} 2^i = 2^{log_3(n)}-1$$

  \medskip

  \textbf{Given} $\epsilon = 1 > 0$: \\
  \begin{align}
  T(n) &= 2T(n/3) + 1 \\
  f(n) = 1 &= O(n^{log_3(2) - \epsilon}) \\
  &= O(1) \\
  \therefore T(n) &= \Theta(n^{log_3(2)})
  \end{align}

  
  \newpage

\item \textit{Professor McGonagall asks you to help her with some
    arrays that are \texttt{raludominular}. A raludominular array has
    the property that the subarray $A[1..i]$ has the property that
    $A[j] > A[j + 1]$ for $1 \le j < i$, and the subarray $A[i..n]$ has
    the property that $A[j] < A[j + 1]$ for $i \le j < n$. Using her
    wand, McGonagall writes the following raludominular array on the
    board $A = [7, 6, 4, -1, -2, -9, -5, -3, 10, 13]$, as an example.}

  \begin{enumerate}
  \item \textit{Write a recursive algorithm that takes asymptotically
      sub-linear time to find the minimum element of A.}
    
    \begin{verbatim}
    def globmin(A, i):
        if (A[i] > A[i+1]):
            return globmin(A, i+1)
        else return A[i]
    \end{verbatim}
    
  \item \textit{Prove that your algorithm is correct. (Hint: prove
      that your algorithm's correctness follows from the correctness
      of another correct algorithm we already know.)}

    Assuming 0-based indexing:
    \medskip

    \textbf{Invariant:} $A[i] < \{x | x \in A[0..i-1]\} \vee A[0..i-1] = \varnothing$ \\

    \textbf{Initialization:} i = 0, so A[0..-1] is the empty set and the loop invariant holds. \\

    \textbf{Maintenance:} if ($A[i] > A[i+1]$), the loop invariant holds and the function calls itself with the next index. Otherwise, it returns A[i], as $A[i-1] > A[i] < A[i+1]$. \\

    \textbf{Termination:} once min() returns: $A[i-1] > A[i] < A[i+1]$, and the loop invariant still holds because i was only incremented if $A[i] < A[i-1]$. This means A[i] is the minimum element in the subarray A[0..i]. Combining this fact with the definition of the 'raludominular' array, A[i] must also be the minimum element in the sub array A[i..n]. Thus A[i] must be the minimum element in A[0..n] = A.

    \newpage
    
  \item \textit{Now consider the \texttt{multi-raludominular}
      generalization, in which the array contains k local minima,
      i.e., it contains k subarrays, each of which is itself a
      raludominular array. Let $k = 2$ and prove that your algorithm
      can fail on such an input.}

      \medskip
    The algorithm will return as soon as it finds an index i such that $$A[i-1] > A[i] < A[i+1]$$ This will be the first local minimum in the first subarray. This will not always be the global minimum, so the algorithm is incorrect for this type of input. 

  \newpage 

  \item \textit{Suppose that $k = 2$ and we can guarantee that neither
      local minimum is closer than $n=3$ positions to the middle of
      the array, and that the ``joining point'' of the two
      singly-raludominular subarrays lays in the middle third of the
      array. Now write an algorithm that returns the minimum element
      of A in sublinear time. Prove that your algorithm is correct,
      give a recurrence relation for its running time, and solve for
      its asymptotic behavior.}

    \begin{verbatim}
    def globmin(A, i):
        while (A[i+1] < A[i]):
          ++i
        if (i < (A.length-(A.length/3))):
          return min(A[i], globmin(A, (A.length-(A.length/3)))
        return A[i]
    \end{verbatim}
    
    For the while loop on the first two lines of globmin(): \\
    Let l = the initial value of i before entering the loop. \\
    Let m = the end of the current raludominular subarray of A \\
    \textbf{Invariant:} $A[i] < \{x | x \in A[0..i-1]\} \vee A[0..i-1] = \varnothing$ \\
    \textbf{Initialization:} i = 0, so A[0..-1] is the empty set and the loop invariant holds. \\
    \textbf{Maintenance:} if ($A[i+1] < A[i]$), i is incremented preserving the loop invariant for the next iteration. \\
    \textbf{Termination:} the loop exits when $A[i+1] \geq A[i]$, and the loop invariant still holds. Furthermore: $A[l..i-1] > A[i] < A[i+1]$. Combining this with the definition of 'raludominular' arrays means A[i] is the local minimum for the current subarray A[l..m]. 

    \medskip

    Now, if we had just found the first local minimum, the if statement would return True since A[i] is necessarily not within the last third of A. From here, the minimum of A[i] (the first local minimum) and the result of a second call to globmin() starting at the last third of A is returned. We just proved that we will find the second minimum in this second call, and the if statement will now return False - simply returning A[i] (now our second minimum) to be compared with our first minimum. The result of this comparison is returned, meaning we have returned the minimum of our two local minima - the global minimum.

    \newpage

    \textit{Running time:} \\
    Because the only costs of this algorithm are the while loop and the if statement evaluation (a constant +1 to each recursion), we really just need to calculate how many times the loop runs to find the cost. The loop only runs for each beginning section A[0..i] of the current raludominular subarray. For the first subarray, this is \underline{at most} $(\frac{n}{2})-(\frac{n}{3})=\frac{n}{6}$ (the local minimum in this subarray cannot be closer than $\frac{n}{3}$ to the center of the array at $\frac{n}{2})$. For the second array, we begin iterating at $\frac{n}{3}$ from the end, so this is \underline{at most} $\frac{n}{3}$. 

    \medskip 

    Thus, the maximum number of times i can be incremented during a complete run is $\frac{n}{6} + \frac{n}{3} = \frac{n}{2}$. \\

    A recurrance relation for this algorithm would be: 
    $$
    T(n) = 
    \begin{cases}
      T(n/3) + \frac{n}{6} + 1, & \text{if}\ n<A.length-(A.length/3) \\
      1, & \text{otherwise}
    \end{cases}
    $$

    However a relation is not needed. Since we know the maximum possible aggregate value ($\frac{n}{2}$) of the driving term, it follows that $T(n) = O(\frac{n}{2} + 2) = O(n)$.\\

  \end{enumerate}

  \newpage
  
\item \textit{Asymptotic relations like $O$, $\Omega$, and $\Theta$ represent
    relationships between functions, and these relationships are
    transitive. That is, if some $f(n) = \Omega(g(n))$, and
    $g(n) = \Omega(h(n))$, then it is also true that
    $f(n) = \Omega(h(n))$. This means that we can sort functions by their
    asymptotic growth.\footnotemark}

  \textit{Sort the following functions by order of asymptotic growth
    such that hte final arrangement of functions
    $g_1,g_2 \dots,g_{12}$ satisfies the ordering constraint
    $g_1 = \Omega(g2)$, $g_2 = \Omega(g_3)$, \dots , $g_{11}=\Omega(g_{12})$}.


  \begin{tabular}{|l|l|l|l|l|l|l|l|l|l|l|l|}
    \hline
    $n$ & $n^{1.5}$ & $8^{\lg n}$ & $4^{\lg* n}$ & $n!$ &$(\lg n)!$ & $(\frac{5}{4})^n$ & $n^{1/ \lg n}$ & $n \lg n$ & $\lg(n!)$ & $e^n$ & $42$ \\
    \hline
  \end{tabular}

  \textit{Give the final sorted list and identify which pair(s)
    functions $f(n)$, $g(n)$, if any, are in the same equivalence
    class, i.e., $f(n) = \Theta(g(n))$.}

  \textbf{From highest order to lowest order:}
  \begin{itemize}
    \item $n!$
    \item $e^n$
    \item $(\frac{5}{4})^n$
    \item $lg(n)!$
    \item $8^{lg(n)}$
    \item $n^{1.5}$
    \item $n$
    \item $4^{lg^*(n)}$
    \item $42$
    \item $n^{\frac{1}{lg(n)}}$
  \end{itemize}

  \begin{align}
    42 = \Theta(x^{\frac{1}{lg(n)}}) 
  \end{align}

\end{enumerate}

\footnotetext{The notion of sorting is entirely general: so long as
  you can define a pairwise comparison operator for a set of objects
  \texttt{S} that is transitive, then you can sort the things in
  \texttt{S}. For instance, for strings, we use a comparison based on
  lexical ordering to sort them. Furthermore, we can use any sorting
  algorithm to sort \texttt{S}, by simply changing the comparison
  operators $>$, $<$, etc. to have a meaning appropriate for
  \texttt{S}. For instance, using $\Omega$, $O$, and $\Theta$, you could apply
  QuickSort or MergeSort to the functions here to obtain the sorted
  list.}

\newpage

\textbf{References Used}

\hrulefill

\begin{enumerate}
  \item CLRS
  \item \url{https://courses.engr.illinois.edu/cs173/fa2009/Lectures/lect_22.pdf}
  \item \url{https://en.wikipedia.org/wiki/Iterated_logarithm}
\end{enumerate}
  

\end{document}
