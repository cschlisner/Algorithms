\documentclass[12pt]{article}
\setlength{\oddsidemargin}{0in}
\setlength{\evensidemargin}{0in}
\setlength{\textwidth}{6.5in}
\setlength{\parindent}{0in}
\setlength{\textwidth}{16cm}
\setlength{\topmargin}{1in}
\addtolength{\topmargin}{-1.5in}
\setlength{\textheight}{23cm}
\setlength{\parskip}{0.75cm}

% Brackets
\usepackage{mathtools}
\DeclarePairedDelimiter\ceil{\lceil}{\rceil}
\DeclarePairedDelimiter\floor{\lfloor}{\rfloor}

% Tikz settings
\usepackage{tikz}
\usetikzlibrary{trees}
\usetikzlibrary {positioning}
\definecolor {mypurple}{cmyk}{0.6,0.4,0.1,0}
\definecolor {myred}{cmyk}{0,0.3,0.3,0}
\usetikzlibrary{fit,shapes.misc}

% Typesetting options
\usepackage{fancyvrb}
\usepackage{amsmath,amsfonts,amssymb}
\usepackage [english]{babel}
\usepackage{url}

\begin{document}

\noindent CSCI 3104 Spring 2018 \hfill Problem Set 6\\
First Last (mm/dd) 

\hrulefill

\fontfamily{cmr}\selectfont

% ******************* PROBLEM 1 *********************
\section*{Problem 1}

\textit{(15 pts) An exhibition is going on in Hogsmeade today from t = 0 (9 am) to t = 720 (6
pm), and world-renowned wizards will attend! There are $n$ wizards $W_1, \cdots, W_n$ and
each wizard $W_j$ will attend during a time interval $I_j:[s_j,e_j]$ where in $0 \le s_j < e_j \le 720$. Note: the ends of the interval are \textbf{inclusive}. The stores in Hogsmeade want to
broadcast magical ads in the sky during the exhibition, multiple times during the day.
In particular, each wizard must see the ad but the store also wants to minimize the
number of times the ad must be shown.}

\textit{For example:}

\begin{center}
\centering
\begin{tabular}{c|c}
  Wizard & [s, e] \\
  \hline
  Minerva McGonagall & [3, 51] \\
  Harry Potter & [6, 60] \\
  Ron Weasley & [6, 99] \\
  Hermione Granger & [105, 155] \\
  Gilderoy Lockhart & [121, 178] \\
  Viktor Krum & [86, 186]
\end{tabular}
\end{center}

\textit{Then, if the ad is shown at times $t_1=51$ and $t_2 = 150$, then all 6 of the wizards will
see the ad.}

\pagebreak

\begin{enumerate}
\item[(a)] \textit{Greedy algorithm $\mathcal{A}$ selects a time instance when the maximum number of wizards are present simultaneously. An ad is scheduled at this time and the wizards who
see this ad are then removed from further consideration. The algorithm A is then
applied recursively to the remaining wizards.}

\textit {Give an example where this algorithm shows more than the minimum number of
ads needed.}
\\\\
\begin{center}
\centering
\begin{tabular}{c|c}
  Wizard & [s, e] \\
  \hline
  1 & [0, 50] \\
  2 & [25, 140] \\
  3 & [25, 100] \\
  4 & [75, 175] \\
  5 & [75, 175] \\
  6 & [150, 250]
\end{tabular}
\end{center}

The greedy algorithm would opt to show 3 ads total: \\
\begin{itemize}
  \item The first choice where the maximum amount of wixards are at the exhibition at the same time (4) would be sometime in the range [75,100]. This removes wizards 2-5 from consideration, leaving wizards 1 and 6. 
  \item The second and third choices would each be one of the intervals during which the remaining two wizards are there - specifically [0,50] and [150,250].  
\end{itemize}


The minimum amount of ads needed for this example is 2 -- one ad at time t=50 would show the ad to wizards 1-3, and another ad at time t=150 shows the ad to the wizards 4-6.

\pagebreak

\item[(b)]\textit{(10 pts extra credit) Let $W_j$ represent the renowned wizard who leaves first and
let $[s_ j,\,e _j]$ be the time interval for $W_j$. Suppose we have some solution $t_1, t_ 2, \cdots, t_k$ for the ad times that cover all wizards. Let $t_1$ be the earliest ad time.}

\textit{Prove the following facts for the earliest scheduled ad (at time $t_1$). For each part, your proof must clearly spell out the argument. Overly long explanations or proofs by examples will receive no credit.}

\textit{S1. Prove $t_ 1 \le e_j$. (Three sentences. Hint: proof by contradiction.)}\\

Suppose $t_1 > e_j$; because $e_j$ = the time at which $W_j$ leaves, $W_j$ would not have seen an ad during his time at the exhibition, as there were no ads shown before $t_1$.  Therefore $t_1, t_ 2, \cdots, t_k$ is not a solution.
\\\\

\textit{S2. If $t_1 < s_j$, then $t_1$ can be deleted, and the remaining ads still form a valid
solution. (Five sentences. Hint: suppose deleting $t_1$ leaves results in a wizard
not seeing the ad; think about when that wizard must have arrived and left
relative to $t_1, s _j, e_ j$. Prove a contradiction.)}
\\

Suppose deleting $t_1$ results in a wizard ($W_0$) not seeing the ad. Because $t_1 < s_j$, $W_0$ must have arrived prior to $W_j$. $W_0$ must also leave after $e_j$, as $e_j$ is the time of the first departure. Thus $[s_ j,\,e _j] \subset [s_0,\,e_0]$, meaning if $W_j$ sees an ad, $W_0$ will see the same ad. Therefore if $W_0$ sees only the ad at $t_1$, then $W_j$ will never be shown an ad during their visit and $t_1, t_ 2, \cdots, t_k$ is not a solution.
\\\\

\textit{S3. If $t_1 < e_j$ , then $t_1$ can be modified to be equal to $e_j$ , while still remaining
a valid solution. (Three sentences. Hint: suppose setting $t _1 := e _j$ leaves a
wizard uncovered--that is, without having seen an ad--then when should that wizard have arrived and left? Prove a contradiction.)}
\\\\
Suppose setting $t _1 := e _j$ means a wizard did not see an ad during their visit. That wizard must have left before $e_j=t_1$, meaning $e_j$ is not the time at which the first wizard left. 

\pagebreak

\item[(c)]\textit{Use the results stated in (1b) to design a greedy algorithm that is optimal.}
  \begin{enumerate}
  \item[(i)] \textit{Write pseudocode for your algorithm.}\\
  input w = list of tuples of the form ($s_i$, $e_i$) where: \\
  $s_i$ = arrival time of wizard i\\
   $e_i$ = departure time of wizard i.
  \begin{verbatim}
  1.  greedyAdTimes(w){
  2.    remaining = w
  3.    ads = []  // list of ad times
  4.    while remaining.length > 0{
  5.      // t = minimum departure time of all remaining wizards
  6.      t = min( [x[1] for x in remaining] ) 
  7.      ads.append(t)
  8.      // remove all wizards that saw ad at time t
  9.      remaining = [x for x in remaining if (x[0] > t or t > x[1])] 
  10.    }
  11.    return ads
  12. }
  \end{verbatim}
  [working python code in appendix 1]

  \pagebreak

  \item[(ii)] \textit{Prove that your algorithm is correct (\textbf{assuming the results stated in (1b)}) and give its running time complexity.}
  \\\\
  My algorithm simply repeats the following steps until every wizard has seen an ad:
  \begin{enumerate}
    \item line 6: find $t_1=e_j$ (as defined in 1b) for the current list of wizards who have yet to see an ad
    \item line 7: add $t_1$ to the set of selected ad times
    \item remove all wizards present at time t=$t_1$ from the list, as they will see an ad at $t_1$. 
  \end{enumerate}

  Essentially, it just chooses an ad time at t=$e_j$ for a subset of attending wizards until every wizard has seen an ad. Because $e_j$=the time at which the first wizard leaves is chosen for each subset of wizards, all wizards will see an ad. \\\\

  % This greedy algorithm is optimal because the problem exhibits optimal substructure in that the optimal solution for a set W of n wizards includes the optimal solution for the set of wizards $V = W-W_1$ (where $W_1$ here is analogous to $W_j$ in 1b). \\\\
  $e_j$ is the optimal choice for the first ad because we know from \textit{S1}: $t_1 \le e_j$ and we know from \textit{S3}: $t_1$ can be set to $e_j$ without affecting the validity of the solution. Setting $t_1$ to $e_j$ maximizes the probability that other wizards (besides $W_j$) will also see the ad. This is the greedy choice. 


  The algorithms main loop runs $\le n$ times - each time:
  \begin{enumerate} 
    \item computing the minimum departure time $e_j$ from the current subset of wizards, which takes O(n) time (although this could be optimized with a priority queue). 
    \item removing the wizards present at time $e_j$, which also takes O(n) 
  \end{enumerate}


  This means the algorithm runs in $O(n^2)$ in the worst case in which no wizards attend simultaneously.  

  \item[(iii)] \textit{Demonstrate the solution your algorithm yields when applied to the $n = 6$
example above.}
 \\\\
 remaining wizards:  [(3, 51), (6, 60), (6, 99), (105, 155), (121, 178), (86, 186)] \\
min departure time = ad 1 =  51 \\
remaining wizards:  [(105, 155), (121, 178), (86, 186)] \\
min departure time = ad 2 =  155 \\
remaining wizards: [] \\
ad times = [51, 155]
  \end{enumerate}

\end{enumerate}

\newpage

% ******************* PROBLEM 2 *********************
\section*{Problem 2}

\textit{(20 pts) Professor Dumbledore needs helpers to watch the gates as much as is possible.
In order to minimize disruption to their class schedules, he asks students and professors
when they are available, and they each provide a set of time ranges. To simplify
scheduling matters, Prof. Dumbledore will simply select a set of these ranges and
assign the relevant people--that is, he never assigns someone just a part of one of their
ranges.}

\textit{Given a set $S$ of $n$ time ranges on a given day, he asks you to find a subset $T$ of these
ranges which is \textbf{covering}, in the sense that every time that could be covered by someone
according to all the ranges $S$, \textbf{is} covered by one of the ranges in $T$. Your goal is to
minimize the size of $T$ (=the \textbf{number} of ranges it contains, regardless of how long they
are).}

\textit{For the following, assume that Dumbledore gives you an input consisting of a single
array $S$ where the $i$-th element S[i] describes the i-th range as a pair $(s_i, l_i)$ where $s_i < l_i$.}

\pagebreak 
\begin{enumerate}
\item[(a)]\textit{In pseudo-code, give a greedy algorithm that computes the minimum-size covering
subset $T$ in $\mathcal{O}(n \log n)$ time. Explain your solution in plain English as well, and
prove an $\mathcal{O}(n \log n)$ upper bound on its running time. (Hint: Start with the earliest
range.)}
\\
\begin{verbatim}
def greedyRanges(r){
    S = r                           // copy input to array S
    u = 0                           // latest uncovered time
    T = []                          // list of selected ranges
    maxl = max([x[1] for x in S])   // maximum extent of ranges
    while (u < maxl) {  
        maxrg=(0,0)      // range that maximizes additional coverage from t <= u
        for rg in S {
            if (rg[0] <= u and rg[1] > u and rg[1] - u > maxrg[1] - u):
                maxrg = rg  // update maximum across all ranges
        }
        if maxrg == (0,0):          // there were no ranges beginning at or before u (a gap)
            min([x[0] for x in S])  // set u to the minimum range start and repeat 
            continue
        u = maxrg[1]        // set u to the end of the maximum range
        T.append(maxrg)     // add the maximum range to the selected ranges
        S.remove(maxrg)     // remove the maximum range from S, to save time
    }
    return T
}
\end{verbatim}

This algorithm optimizes for the largest amount of new time covered beginning from the end of the previous covered time. \\
Specifically it chooses the range r $\in$ S such that $l_r - u$ is maximal given $s_r <= u$, where u is the latest time that is not covered by a selected range. 



\item[(b)]\textit{Prove that your algorithm is correct, in particular, that it correctly computes the \textbf{minimum-size} covering subset.}
\\\\
TODO
\\

\end{enumerate}

\newpage

% ******************* PROBLEM 3 *********************
\section*{Problem 3}

\textit{(20 pts) We saw on the previous problem set that the cashier’s (greedy) algorithm
for making change doesn’t handle arbitrary denominations optimally. In this problem
you’ll develop a dynamic programming solution which does, but with a slight twist.
Suppose we have at our disposal an arbitrary number of \textbf{cursed} coins of each denomination $d_1 , d_2, \cdots, d_k$, with $d_1 < d_2 < \cdots < d_k$, and we need to provide n cents in change.
We will always have $d_1 = 1$, so that we are assured we can make change for any value
of $n$. The curse on the coins is that in any one exchange between people, with the
exception of $i = 2$, if coins of denomination $d_i$ are used, then coins of denomination
$d_{i-1}$ cannot be used. Our goal is to make change using the minimal number of these
cursed coins (in a single exchange, i.e., the curse applies).}

\begin{enumerate}
\item[(a)]\textit{For $i \in \{1, \cdots, k \}$, $n \in \mathbb{N}$, and $b \in \{0,1\}$, let $C(i, n, b)$ denote the number of cursed coins needed to make $n$ cents in change using only the first $i$ denominations
$d_1, d_2, \cdots, d_i$, where $d_{i-1}$ is allowed to be used if and only if $i \le 2$ or $b = 0$. That is, b is a Boolean “flag” variable indicating whether we are excluding denomination
$d_{i-1}$ or not ($b = 1$ means exclude it). Write down a recurrence relation for $C$ and
prove it is correct. Be sure to include the base case.}
\\\\
TODO
\\
\item[(b)]\textit{Based on your recurrence relation, describe the order in which a dynamic programming table for $C(i, n, b)$ should be filled in.}
\\\\
TODO
\\
\item[(c)]\textit{Based on your description in part (b), write down pseudocode for a dynamic
programming solution to this problem, and give a $\Theta$ bound on its running time
(remember, this requires proving both an upper and a lower bound).}
\\\\
TODO

\end{enumerate}

% ---------------------------------------------------

\newpage

\textbf{References} \\
\hrulefill
\begin{enumerate}
  \item CLRS
\end{enumerate}

\newpage

\textbf{Appendix 1} \\
\hrulefill
Python 3 code for (1c): \\
\begin{verbatim}
  def greedyAdTimes(w):
    remaining = list(w)
    ads = []
    while len(remaining) > 0:
      ej = (min(remaining, key=lambda x:x[1]))[1]
      ads.append(ej)
      remaining = list(filter(lambda x: x[0] > ej or ej > x[1], remaining))
    return ads
\end{verbatim}


\end{document}